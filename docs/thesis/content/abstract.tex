\pagenumbering{roman}
\setcounter{page}{1}

\selecthungarian

%----------------------------------------------------------------------------
% Abstract in Hungarian
%----------------------------------------------------------------------------
\chapter*{Kivonat}\addcontentsline{toc}{chapter}{Kivonat}

A diplomatervezési feladat egy, a tanszéken futó kutatási projekthez kapcsolódik, amelyben egy intelligens autó tesztplatform létrehozása a cél. A platform alapja egy csökkentett méretű (kb. 1:3 méretarányú) autómodell, melyet egy távirányítható játékautóból alakítunk át. A tesztplatform létrehozásának célja különböző navigációs algoritmusok fejlesztésének és valós környezetben történő tesztelésének elősegítése.

A sikeres navigáció és az ütközések elkerülése érdekében fontos, hogy a jármű valós időben detektálni tudja a környező objektumokat, és megfelelően reagáljon rájuk. Ennek elősegítése érdekében az autóplatformon elhelyezésre került egy‑egy vízszintes síkban pásztázó lézeres távolságszkenner (lidar) a jármű elején és hátulján.

A feladat keretében ezekre a szenzorokra építve kell megvalósítani mozgó objektumok felismerését, azok méretének és sebességvektorának becslésével együtt. Fontos, hogy a jármű el tudja különíteni a statikus és a mozgó akadályokat egymástól. További feladat egy olyan akadályelkerülési módszer megvalósítása az autón, amely az objektumok mozgását is figyelembe véve hozza meg a megfelelő navigációs döntést minden időpillanatban.

A megvalósított algoritmusok működését mind szimulált, mind valós környezetben szükséges ellenőrizni. A valós tesztelés történhet a tanszéki járműplatformon, vagy egy kisméretű, egyedileg felépített modellautón is.


\vfill
\selectenglish


%----------------------------------------------------------------------------
% Abstract in English
%----------------------------------------------------------------------------
\chapter*{Abstract}\addcontentsline{toc}{chapter}{Abstract}

The theme of the thesis is a sub-task of a research project at the Department of Automation and Applied Informatics, with the purpose of designing an intelligent car testing platform. The platform is based on a decreased-size (around 1:3 scale) remote control car model, that has been modified to support self-driving program control. The platform aims to support the development of different navigation algorithms and helping the testing in real environment.

For a successful navigation and obstacle avoidance, the detection of the surrounding objects and reacting accordingly are essential. For the purpose of detection, two horizontal distance scanning sensors (LIDARs) have been placed on the car - one on the front and one on the back.

Part of the task is to implement the detection of the moving obstacles and estimate their sizes and speed vectors. It is important for the car to be able to separate static and moving objects from each other. The other part is developing an obstacle-avoidance method, implemented as an application on the car, that takes the moving objects into consideration while making navigational decisions at every time step.

The implemented algorithms need to be tested both in simulation and in real environment, that may be executed on  the department's vehicle platform or on a small-scale, custom-build model car.

\vfill
\selectthesislanguage

\newcounter{romanPage}
\setcounter{romanPage}{\value{page}}
\stepcounter{romanPage}\textbf{}