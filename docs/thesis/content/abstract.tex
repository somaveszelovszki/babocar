\pagenumbering{roman}
\setcounter{page}{1}

\selecthungarian

%----------------------------------------------------------------------------
% Abstract in Hungarian
%----------------------------------------------------------------------------
\chapter*{Kivonat}\addcontentsline{toc}{chapter}{Kivonat}

A diplomatervezési feladat egy, a tanszéken futó kutatási projekthez kapcsolódik, amelyben egy intelligens autó tesztplatform létrehozása a cél. A platform alapja egy csökkentett méretű (kb. 1:3 méretarányú) autómodell, melyet egy távirányítható játékautóból alakítunk át. A tesztplatform létrehozásának célja különböző navigációs algoritmusok fejlesztésének és valós környezetben történő tesztelésének elősegítése.

Az alábbi diplomaterv két fontos funkció megvalósítását részletezi, melyek hozzásegítik az autót a sikeres térképezéshez és navigációhoz.

Az egyik funkció egy olyan térképalkotási módszer, mely képes különválasztani a térben érzékelt statikus (nem mozgó) akadályokat a dinamikus (mozgó) objektumoktól. A dolgozat során részletezésre kerül mind a módszer elméleti bemutatása, az algoritmus be- és kimenő adattípusainak és belső működésének részletezése, illetve a szimulációban és a valós autón elvégzett tesztek eredmények kiértékelése.

A másik funkció az előbb említett térképalkotásra épül. Ennek a feladata a térben elhelyezkedő álló és mozgó objektumok közötti lokális akadályelkerülés. Ez a gyakorlatban egy előre megadott trajektória tartását jelenti, anélkül, hogy bármely akadálynak nekiütközne az autó. A térképezéshez hasonlóan ennek a funkciónak is részletes leírásra kerül mind a működése, mind az algoritmust bemutató tesztesetek.

\vfill
\selectenglish

%----------------------------------------------------------------------------
% Abstract in English
%----------------------------------------------------------------------------
\chapter*{Abstract}\addcontentsline{toc}{chapter}{Abstract}

The theme of the thesis is a sub-task of a research project at the Department of Automation and Applied Informatics, with the purpose of designing an intelligent car testing platform. The platform is based on a decreased-size (around 1:3 scale) remote control car model, that has been modified to support self-driving program control. The platform aims to support the development of different navigation algorithms and helping the testing in real environment.

The following thesis describes two important features, that help the car perform mapping and navigation.

One of these features is a mapping algorithm, that is able to separate the static (non-moving) obstacles from the dynamic (moving) objects. During the thesis, the theoretical background, the input and output parameters and the inner operation of the algorithm are described, along with the evaluation of the tests performed in simulation and also with the real car.

The other feature is built on the upper mentioned mapping task. This algorithm performs local obstacle avoidance based on the static and dynamic objects in the map. In practice, this means that the car needs to keep a pre-defined trajectory, without colliding with any obstacles. Similar to the map-building, this feature is also explained in detail, and test cases are also available.

\vfill
\selectthesislanguage

\newcounter{romanPage}
\setcounter{romanPage}{\value{page}}
\stepcounter{romanPage}\textbf{}