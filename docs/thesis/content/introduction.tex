%----------------------------------------------------------------------------
\chapter{\bevezetes}
%----------------------------------------------------------------------------

Nowadays, self-driving cars gain more and more attention, both their technology and their effect on people's daily routines and their lives overall. Several articles are published every year about how these cars will change the way people commute to work, visit their friends or go on a family vacation. These articles often point out the decrease of the number of accidents, an optimized load of traffic and thus a reduced fuel consumption as the major advantages of this technology-to-come.

The release date of these cars, however, is still a matter of question. In 2015, Mark Fields, president and CEO of Ford at the time estimated their first fully autonomous car in 2020. 2 years later, at CES 2017 Nvidia announced that with the partnership of Audi they would develop a self-driving vehicle - also, in market by 2020. Both of these statements are considered too ambitious guesses today, as there is a high probability that we need to wait until at least 2025 for reliable fully autonomous vehicles to hit the roads. Claiming that there are no viable signs of these cars in traffic would be a false statement, as there are several companies who have been testing there vehicles on public roads for the last years, but these prototypes are very far from reliable products yet. The company that seems to be ahead of the competition in this race is Tesla. Their self-drving software is already in their products, but it still needs millions of hours of testing and the responsibility is still the driver's if an accident happens.

But why is this delay of release dates? One possible answer is that manufacturerers have the tendency to exaggerate when asked about new products, and thus the users' need and the other competitors development speed urged them to make such estimations they could not keep up with. Another theory is that the companies acknowledge how many hours and kilometers of testing it needed to finalize a self-driving product. The moment of realization may have come for many developers in the form of an accident

However, the spreading of autonomous vehicles is blocked by several legislational and technological obstacles. As this thesis describes an engineering problem an its solution, I will reflect on the technological blockers that both car manufacturers and other self-driving software developer companies need to face. Just to list some of these problems, we can name reliable object detecion, error insensitive, robust decision-making, fast, well-tuned physical control, and to meet the safety and quality requirements set by the market, determinant scheduling and redundancy throughout the whole software are also essential.

 